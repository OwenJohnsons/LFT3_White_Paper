Humanity is at a tipping point for conducting effective radio astronomical observations from earth. The ever-ncreasing use of wireless communication devices on the ground and in orbit means that nowhere on our planet, even  remote places with very low population density, is free of significant RFI. We have already gone to space with a small number of very expensive “Great Observatories,” but advent of accessible and relatively cost-effective access to space means that deploying more and smaller telescopes is now feasible. We are therefore at the tip of the spear of the transition from earth-based to space-based telescopes for radio astronomy---born of necessity to respond to changes in earth’s RF environment from humanities' activities.\footnote{Note that other wavelengths have always needed to be above the atmosphere due to absoprtion and scattering by atmospheric constituents.}  

While terrestrial wireless communications has strictly negative implications for radio astronomy, cost-effective access to space means that smaller telescopes and experiments may be deployed to areas with significantly less RFI. As this happens, the sensors themselves along with other active transmitters in space will begin injecting RFI into those domains. This makes it urgent to to get well-designed sensors into space now to make early baseline measurements and conduct the unique science allowed by the current essentially complete lack of RFI. The lunar farside is unique in our solar system in that it always has its back turned to the Earth \cite{heidmann2002,MACCONE2019233,michaud2020lunar}. The lunar farside presents a once-in-human-history opportunity to record signals in a fully quiet environment. 

To act on this singular opportunity, we propose a telescope to be landed near the lunar antipode within five years to conduct these unique-in-history measurements.  The telescope will comprise a UHF dual-pol multibeam phased array operating from 300--900 MHz as well as individual elements covering 1--50 MHz, 50--100 MHz and 600--1800 MHz.  During its mission lifetime, the telescope will observe most of the lunar farside sky and conduct historical surveys in the most RFI-pristine environment in the solar system. 